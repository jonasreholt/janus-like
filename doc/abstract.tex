\section*{Abstract}
A reversible programming language forces the programmer to write reversible programs, i.e.\
a program that can run in both directions; computing both from input to output, and output to
input. Reversible languages are interesting to study, as it removes the theoretical lower limit
of heat generation, and that one only needs to write a single program, to handle functionality
that has a natural inverse, e.g.\ encryption. One downside to reversible language is, the added
overhead needed to guarantee correct behavior in both directions.

This thesis presents a optimizing module aimed at removing this runtime overhead. As the overhead
stem from runtime assertions, the presented optimizer tries to prove these assertions to always
hold using the satisfiability modulo  theory solver \texttt{z3}. To do this the thesis first
presents a simple imperative reversible language, similar to \texttt{Janus}. The optimizer
translates from this language's abstract syntax tree into a \texttt{z3} model, which is then
queried at every program point, that involves a runtime assertion. This thesis addresses some
of the complexion of this translation and querying to \texttt{z3}, when going from a reversible
language, e.g.\ the user defined possibly bogus runtime assertions from the \lsin{fi} associated
to if statements.

This project shows that using the SMT solver \texttt{z3} to analyze reversible programs, in order
to remove the runtime overhead stemming from reversibility properties, is a doable. The SMT solver
utilizes the fact, that most assertions should always be true, unless the programmer wrote bogus
assertion, and is able to remove a significant amount of assertions. The biggest obstacle is
achieving the right amount of information to prove this validity.

This project showcase a complete translation from the \texttt{Janus} like language \lan to
\texttt{C++}, with a web interface for a broader field of application. This method could be useful
with further investigation into decreasing compile time, and utilizing statically known information
better, as these are the two main difficulties found during this project.