\section{Conclusion and future work }
This project presents the language \lan derived from the reversible language \texttt{Janus}.
\lan keeps the reversibility property derived from \texttt{Janus}, but changes e.g.\ the loop
construct in order to remove some runtime assertions directly from the language construct.

The \lan compiler presented in this project, translates from the source language into \texttt{C++}
code. It tries to remove runtime assertions by the use of the SMT solver \texttt{z3}. This is
done by recursing through the AST, translating into a \texttt{z3} model using its API for
\texttt{Haskell}. Each time a construct creating a runtime assertions is met, a query is made to
\texttt{z3}, in order to determine whether the assertion can be proven to always hold. This is
done by the steps:
\begin{enumerate}
      \item If it is a user-generated assertion, the assertion is checked to be satisfiable.
      \item The assertion is negated, and the \texttt{z3} model is checked to be satisfiable.
      \item If the negated assertion is not satisfiable, it means the assertion is always valid
            and can be optimized away.
\end{enumerate}
\noindent
In order to create a proper translation into \texttt{z3}, a simple type checker goes through the
AST before the optimization phase, both in order to (of cause) type check, but more importantly
annotate the AST with both types and array sizes.
\\
\\
Using the SMT solver \texttt{z3} the \lan compiler showed ability, to optimize most runtime
assertions away. The area in which the chosen method proved to be less effective, was when
analyzing reversed procedures. This probably stems from the fact, that the reversed procedure
some times work with "unspoken" starting contracts e.g.\ that the input numbers are always equal
when the forward procedure has been executed before.

Loop constructs also showed resistance towards optimizations, which makes sense, as \texttt{z3}
has no theory concerning loops. This means measures needed to be taken, in order to gather as much
information as safely possible, for \texttt{z3} to correctly validate assertions. In the best case
scenario loops were unrollable, so the pure information was retrievable, but in all other
cases, generalization methods was needed, to remove any possibly wrong information from the
SMT model.

This project showed great promise in regards to optimizing assertions away from imperative
reversible languages, but at present moment, the execution time of the compiler is rather slow.
For a simple Fibonacci program consisting of $21$ lines of code, the compile time was
$6.2$ seconds, with optimizations. This slow speed stem from the use of the \texttt{z3}
solver, and is probably caused by bad usage by the writer of this bachelor thesis.
\\
\\
The \lan language, as presented in this project, is rather simple, with simple types and
no recursion. For future work, implementing recursion would make sense, in crating a more
modern and pleasant language to work with. The slow execution time of the compiler should,
however, be a top priority, to improve development iteration time, to actually make the
optimization useful, as a "sparring" partner, to improve speed of developed code. As previously
mentioned the focus for this, should probably start at the use of \texttt{z3}, mainly of which
theories are called with the solver.

To avoid removing previously known information about variables, when analyzing loops that cannot
be unrolled, a method somewhat like the one presented in \cite{ai}, where loops are analyzed
using a combination of abstract interpretation and a SMT. This might be able to improve the
analysis of loops.


\subsection{Reflection over project }
After having completed this project, and experiencing the consequences of the workflow I
adopted, I can see that automated tests should probably have been implemented from the start.
The test section also reveals, the lack of unit tests for each function, as only the full
behavior of the compiler is tested. I fell in the trap, of ignoring setting up these tests
using \texttt{cabal}, as I simply felt swarmed already having to learn \texttt{Haskell}, \texttt{z3},
and all the other things needed for this thesis. I probably shouldn't have ignored it, as it lead
to some annoying bugs, when I finally implemented the tests of the full behavior.

As an extension to this, it would probably have been nice to work with more people on this
project, so I had someone to spar with on a daily basis.
\\
\\
Looking back it would probably have made sense to use \texttt{Boogie 2} instead of directly
communicating with \texttt{z3}. \texttt{Boogie 2} is a intermediate language representation, used
for verifying programming languages. Hence it works for multiple SMT solvers, and is used in
projects such as \texttt{Dafny}. Working with this intermediate language, might have made
it easier correctly modelling program structures. The reason why I did not chose it was, that
\lan is a reversible language, meaning it does not follow the ordinary program structure, meaning
representing the reversible semantics in the intermediate code might be unnecessary complicated.