\section{Benchmark of compiled code}
To asses the impact of the optimization run done by the \lan compiler, the
example programs presented in the we interface, except for the "unrollableLoop" program.
However, extra loops calling and uncalling each function are inserted to make the programs
more demanding, creating hot loops
(see \ref{sec:benchmark-programs}). Two factors are used to asses the impact: 1) compile time
of program, and 2) running time of compiled program. For the latter the output \texttt{C++}
program is translated to machine code using the \texttt{g++} compiler, version $9.4.0$
for \texttt{Ubuntu}. No extra optimizations is done using the \texttt{g++} compiler, to
isolate the pure impact of the \lan compiler optimization.
\\
\\
To benchmark compile time the \texttt{Criterion} \texttt{Haskell} package has been used,
and the benchmark is build using \texttt{Cabal}. Because the \lan compiler utilizes
standard out and standard error, to output its result, a small command is written in the
\texttt{Makefile} to suppress information that is not needed for the benchmark. Hence
to benchmark the compile time with and without optimizations, use the given
\texttt{Makefile} using the command:
$$\texttt{make bench}$$
\noindent
Benchmarking the runtime of the output program is done using the supplied \texttt{bash}
script \texttt{runbenchmark.sh} (see appendix \ref{lst:runbenchmark}). This simply compiles
the \lan compiler to ensure newest edition, compiles executable for each benchmark program,
and then executes each 10 times and reports the average.
\\
\\
The results of the benchmark can be seen in table \ref{table:benchmark-results}.
It should be noted that binary size is calculated using lines of assembly, and not the
actual size in KB of the binary file. This is done, as number of assembly lines is a
more direct measure of the \lan compiler optimizations. The difference is calculated as
the percentage increase from no optimization to optimizations.

\begin{table}[H]
    \centering
    \begin{tabular}{|l||r|r|r||r|r|r||r|r|r|}
        \hline
        \multirow{2}{*}{Program}& \multicolumn{2}{c|}{Compile Time (ms)} & \multirow{2}{*}{Diff (\%)} & \multicolumn{2}{c|}{Runtime (ms)} & \multirow{2}{*}{Diff (\%)} & \multicolumn{2}{c|}{Binary Size} & \multirow{2}{*}{Diff (\%)} \\
        \cline{2-3} \cline{5-6} \cline{8-9}
        & \multicolumn{1}{c|}{(noOpt)} & \multicolumn{1}{c|}{(opt)} & & \multicolumn{1}{c|}{(noOpt)} & \multicolumn{1}{c|}{(opt)} & & \multicolumn{1}{c|}{(noOpt)} & \multicolumn{1}{c|}{(opt)} &  \\
        \hline
        Fibonacci    & $0.6526$ & $5741.00$ & $879611.92$ & $4.79376$ & $4.080000$ & $-14.89$ & $249$ & $171$ & $-31.33$ \\
        \hline
        Factorial    & $0.3725$ & $119.70$  & $32034.23$  & $2.08916$ & $1.81717$  & $-13.02$   & $211$ & $160$ & $-24.17$ \\
        \hline
        Perm to code & $0.6481$ & $70.25$   & $10739.38$  & $1.81760$ & $1.74201$  & $-4.16$  & $264$ & $192$ & $-27.27$ \\
        \hline
    \end{tabular}
    \caption{Results from benchmark. Diff is calculated as percentage increase from no optimizations
    to optimizations enabled.}
    \label{table:benchmark-results}
\end{table}
\noindent
So on average, for the three programs, the \lan compiler is able decrease the amount of
assembly lines by $27.76\%$. Which also shows in the fact, that all programs run faster with
optimization on, than without. Listing \ref{lst:benchmark1} shows the \texttt{C++} code of
running with optimizations and none. This shows that the optimizer was able to remove all
assertions inserted by the reversibility of \lan for the Fibonacci program. The fact that
the runtime of the Fibonacci program benefits the most from the optimizations, stems simply
from its deeper loop structures. Meaning more amount of iterations, the more important are
the optimizations.
\\
\\
The compile time for all three programs, are increased significantly by performing optimizations.
On average the compile time increases with $354367.49\%$, which does lengthen iteration time
when developing software. Furthermore, all three programs are short and relatively simple,
meaning for more complex programs, the compile time difference might increase further.
Therefore, it is not advisable to use optimization during development, but wait for the
release build.
\\
\\
Further work should include focus on developer iteration time instead of code generations,
to test the viability of these optimizations in a developer environment. Profiling the execution
of the compilation shows, that most time is spend in the optimizer, meaning a focus could be
on focusing the optimization for hot code segments, thereby keeping most runtime gains while
hopefully decreasing execution time. Actually, for the Fibonacci program, approximately
$32\%$ of the time is spend on optimizing forward directional part of the program, and
approximately $69\%$ is used on optimizing the backward directional part of the program.
This happens even though the reverse actually skips optimizations for the \lsin{main}
procedure. Hence if the reason for this increase could be found, a good chunk of the
execution might be removable. The fact that the reversed fib is significantly slower
to optimized than the non-reversed, might hint that the \texttt{z3} solver has a harder
time picking the appropriate solvers, based on the information it is given, so it might
end up picking a solver, that directs it in a less efficient direction of the solver-tree.

\begin{varwidth}[t]{0.45\textwidth}
\begin{lstlisting}[language=C++, caption={Optimized Fibonacci program.}]
...
// Global variables defining starting state
unsigned int x1 = 0;
unsigned int x2 = 1;
 
...

void fib_forward(unsigned int &x1, unsigned int &x2, const unsigned int n)
{
    unsigned int i = 1;
    while(i != n)
    {
        unsigned int tmp = (x2) - (x1);
        x2 += x1;
        x1 += tmp;
        i += 1;
    }
}
 
void fib_reverse(unsigned int &x1, unsigned int &x2, const unsigned int n)
{
    unsigned int i = n;
    while(i != 1)
    {
        i -= 1;
        unsigned int tmp = (x1) - ((x2) - (x1));
        x1 -= tmp;
        x2 -= x1;
    }
}
 
int main()
{
    unsigned int i = 0;
    while(i != 1000)
    {
        fib_forward(x1, x2, 200);
        fib_reverse(x1, x2, 200);
        i += 1;
    }
}
\end{lstlisting}
\end{varwidth}
\hspace{4em}
\begin{varwidth}[t]{0.45\textwidth}
\begin{lstlisting}[language=C++, caption={Unoptimized Fibonacci program.}]
... 
// Global variables defining starting state
unsigned int x1 = 0;
unsigned int x2 = 1;
 
...
 
void fib_forward(unsigned int &x1, unsigned int &x2, const unsigned int n)
{
    unsigned int i = 1;
    while(i != n)
    {
        unsigned int tmp = (x2) - (x1);
        x2 += x1;
        x1 += tmp;
        assert(tmp == (x1) - ((x2) - (x1)));
        i += 1;
        assert(!(i == 1));
    }
}
 
void fib_reverse(unsigned int &x1, unsigned int &x2, const unsigned int n)
{
    unsigned int i = n;
    while(i != 1)
    {
        i -= 1;
        unsigned int tmp = (x1) - ((x2) - (x1));
        x1 -= tmp;
        x2 -= x1;
        assert(tmp == (x2) - (x1));
        assert(!(i == n));
    }
}
 
int main()
{
    unsigned int i = 0;
    while(i != 1000)
    {
        fib_forward(x1, x2, 200);
        fib_reverse(x1, x2, 200);
        i += 1;
        assert(!(i == 0));
    }
}
\end{lstlisting}
\end{varwidth}