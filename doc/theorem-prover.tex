% This section should discuss the theory behind the optimizer

\section{Theorem prover } \label{sec:z3}
% Find representative set of theorem provers, and analyze in context of project
This project involve statically analyzing a given program, to identify assertions that are
provably always correct, and therefore redundant. To achieve this, one could look to tools used in
the field of program verification or static analyses for compiler.
Some of these method include abstract interpretation, and the use of SMT
(Satisfiability Modulo Theories) solvers.
\\
\\
Abstract interpretation involves transforming language constructs into a sound approximation,
that can then be partially executed, to assess certain properties of the language construct.
With this, one can withdraw information about the possible executions of the construct, and
thereby get information, that would otherwise not be possible. Hence it can be used to determine,
whether runtime assertions are true or not. However, due to this abstraction of the program,
information is also lost \cite{ai}. A project that already include use of abstract interpretation,
used for a similar use case as this project is \texttt{Astrée} \cite{astree}, which is a static
analyzer that proves the absence of runtime errors. This also includes "user-provided assertions"
\cite{astree}. Hence abstract interpretation has been proven, as a technique for validating
runtime assertions.
\\
\\
SMT solvers see their use in computer aided verification of computer programs, where programs
are transformed into logical models, that can then be verified using a SMT solver. Thereby proving
certain properties of the underlying program \cite{sigda}. For this to work, it is needed to
translate a program from \lan into these mathematical models, that a SMT can then solve. Hence
the work will lay on the translation, and on what information safely can be used for the model.
Because of this mathematical model, SMT solvers work on SSA (Static Single Assignment) form,
meaning during (or before) this translation into a mathematical model, the \lan program must also
be translated into SSA form. A project that use SMT solver to verify properties of programs is
\texttt{Dafny}, a language created with build in program verifications \cite{dafny}. The verifier
for \texttt{Dafny} works as part of the compiler. Hence this method has also been showcased to work
for approximately the same purpose as this project aims to solve.
\\
\\
This project will use the SMT technique to statically analyze \lan programs. This is done because
of the following reasons:

\begin{itemize}
    \item There exists industry-proven solvers on the market, that are fairly straight forward
          to use; many of which are open source projects. Where abstract interpretation requires
          more upfront work, to decide on
          abstraction level sets, and transformation. Hence the SMT solvers has a lower level
          of entrance.

    \item The case of \texttt{Dafny} shows, that it can be a viable solution within the compilation
          process.
\end{itemize}


\subsection{Choosing theorem prover }
The \texttt{SMT-LIB} standard has worked towards standardizing the interface of SMT solvers \cite{smtlib}.
This standardization include a language constructed around S-expressions, best known from
the programming language \texttt{Lisp} \cite{pld}. Following this standardization has at least
the advantage, that the libraries implementing it, have access to a wide variety of benchmark tests,
making their speed directly comparable with other SMT solvers \cite{smtlib}. So narrowing possible
SMT solver to only those that implement the \texttt{SMT-LIB}, gives a possibility to use the
yearly competition \texttt{SMT-COMP}, as one guideline to compare the different solvers.
\\
\\
The first narrowing comes from the supported theories. To use it for \lan, the SMT solver must
support the following theories:

\begin{itemize}
    \item Bitvector theory as \lan works on unsigned 32 bit integers, meaning the value
          representation is bounded, so using normal integer theories would not work.

    \item Boolean theory simply because \lan has boolean values. However, this constraint is
          trivial as SMT is a generalization of the boolean satisfiability problem (SAT), so
          all SMT solvers will support this theory.

    \item Array theory because \lan supports the use of arrays. This theory can be avoided, if
          one simulated arrays by simply having a unique variable for each array element.

    \item Linear arithmetic theory so arithmetic operations can be directly modelled. Non-linear
          arithmetic is not needed, as \lan does not support floats.
\end{itemize}
\noindent
The next narrowing comes from looking at \texttt{SMT-COMP} for the year 2021 \cite{smtcomp}, and see
how each theorem prover performs. Looking through the best performing SMT solvers on the theories
mentioned above three solvers stand out: \texttt{cvc5}, \texttt{z3}, and \texttt{Yices2}.
\texttt{cvc5} is the top performer on the incremental track \cite{smtcomincr} followed by
\texttt{z3}. While \texttt{Yices2} performs good on quantifier-free logic \cite{smtcommod},
where \texttt{z3} also make it to top five in all categories.
\\
\\
Based on the above, the available languages with API's, and the fact that \texttt{z3} is used in
a variety of projects, some of which are similar use cases as this such as \texttt{Dafny},
\texttt{z3} has been chosen as the theorem prover to use for this project. It supports the
needed theories, to make life easy for analyzing \lan, an it performs quite well for these
theories. It s open sourced, and provides API's for the biggest number of languages of the three.
Also \texttt{z3} is a Microsoft product, meaning a big company is backing the continued
progress of the project. Last but not least, a unofficial API for \texttt{z3} is provided
through the programming language \texttt{Haskell}, which I have chosen based on the arguments
in section \ref{sec:z3interaction}.


\subsection{Using the theorem prover }
\texttt{Z3} is a SMT solver created by Microsoft specifically developed
for program verification and analysis \cite{z3:microsoft}. It is therefore geared towards
analyzing whether certain properties of a program are satisfied. In this projects it is to be
used to validate certain properties. Hence the properties will be inverted, and then checked for
satisfiability. If the inverse is not satisfiable then it must be the case, that the property
is always satisfied.

This does however pose a problem, as the project is about validating, and possibly removing,
user created assertion, meaning they can be fallacies. If the user created a fallacy, the inverse
would be satisfiable, meaning the assertion would not be removed. This is in itself fine, but if
the information from the fallacy is carried on to the validation of other assertions, \texttt{z3}
will always report that the model is unsatisfiable, meaning all assertions coming after the fallacy
will, possibly, faultily be removed. E.g. in the simple program snippet below, \texttt{z3} will
believe the assertion arising from \lsin{dealloc int i = 3}, will be faultily deemed valid, as the
inverse is never satisfiable because of the assertion created.

\begin{lstlisting}
    assert(1 == 2)
    local int i = 2
    dealloc int i = 3
\end{lstlisting}
\noindent
To accommodate this, every user inputted information (in the form of deallocations, assertions,
fi-conditions, and invariants) are first checked if satisfiable in their regular form. If this is
the case, it s not a fallacy, meaning the information is okay to carry on in the model.
\\
\\
Every value in \texttt{z3} is immutable, but \lan functions by mutable variables, so a \lan program
must be converted into Static Single Assignment (SSA) form. So a direct translation into
\texttt{z3} is possible. The method proposed by Matthias Braun et al. in \cite{SSA}, will be used
in this project, as it allows for SSA-based optimizations during the construction phase, meaning
the abstract syntax tree, can be optimized by \texttt{z3}, while the transformation into SSA is
being performed.
% TODO: Explain how the algorithm work

\subsubsection{Dealing with \lan constructs translation to \texttt{z3}} \label{sec:dealingWithz3}
\label{translation-to-z3}
To deal with the above mentioned problems with translation from the reversible language \lan
utilizing mutable variables into the logic language \texttt{z3} that works only with immutable
variables, I have used the following rules of translation.
\\
\\
\textbf{Moderation statements}: The problem with this construct is that all variables in
\texttt{z3} are immutable. This can be addressed by creating new fresh variables, and asserting
these fresh variables to the moderated value e.g.
\begin{verbatim}
    x += 5  =>  (declare-const x1 Int)
                (assert (= x1 (+ x 5)))
\end{verbatim}
\noindent
This method mimics the SSA algorithm \cite{SSA}, so the SSA form is only within \texttt{z3}, and
not present in the abstract syntax tree.
\\
\\
\textbf{Conditional statements}:
If the assertion being validated is not associated to a specific conditional, the program will
not know which path is going to be taken at run time. Therefore both paths must be constructed,
and then the \texttt{ite} function from \texttt{z3} can be used to determine the path. e.g.
\begin{verbatim}
    local int x = 5         (define-fun x () Int 5)
    if (x < 5)              (declare-const x1 Int)
    {                   =>  (assert (= x1 (- x 5)))
        x -= 5              (declare-const x2 Int)
    } fi (x == 0)           (assert (= x2 (ite (< x 5) x x1)))
    delocal int x == 5      (assert not (= x2 5))
\end{verbatim}
\noindent
\textbf{Loops}:
Because \texttt{z3} does not have a construction for loops, loops can only be validated by unrolling
them. The question of how far to unroll loops is however rather complex. e.g. what should be done
with non-terminating loops, and how to detect these? Also how far should loops be unrolled? One
important factor is, that the optimizer must be conservative, so to not change the runtime behavior.
Which means if it's not possible to analyze how many times to unroll, the optimization is blocked
by loops.

When it comes to \texttt{for}-loops the analysis is relatively simple, as the constructs both
tell a starting value, what happens to this variable at each iteration, and when to stop.
So as long as the local variable being declared in the beginning, is the same as the one being
deallocated after the loop, and the same being moderated at each iteration, it is straight forward.
However if one of these are not the case, the analysis becomes harder.

The unrolling of a simple \texttt{for}-loop can be done by transforming it into a series of
\texttt{if}-statements that can be represented in \texttt{z3}:
\begin{verbatim}
    local Int bit = 1                   local int bit = 1
    for int i = 0                       local int i = 0
    {                                   if (!(i == 10))
        local int z = bit           =>  {
        bit += z                            local int z = bit   
        delocal int z = bit / 2             bit += z
    } i += 1; untill (int i = 10)           delocal int z = bit / 2
                                            i += 1
                                            if (!(i == 10))
                                            {
                                                ...
                                            } fi (i == 10)
                                        } fi (i == 10)
                                        delocal int i = 10
                                        delocal int bit = 2048
\end{verbatim}
\noindent
\textbf{Invariants}:
Modelling invariants in \texttt{z3} is done as simply, as "inserting" assertions at the appropriate
places of the program. These include at loop initialization, at the start of each iteration, and
just after the loop termination.
\\
\\
\textbf{Function Calls}:
% TODO: Talk about recursion!!!
To model the functions correctly in \texttt{z3} inlining must be done, as all changes done
by a function is through side effects. After inlining the function, it can be modeled using the
other strategies outlined above. This can be done trivially as all procedures are non-recursive;
there can however be a problem if two procedures call each other endlessly. This case is not handled
meaning the optimizer will loop forever.
\\
\\
\textbf{Reversibility}:
Because the translation is from a reversible language to a irreversible language, every procedure
will be translated into two: One going forward, and one backwards. It is necessary to perform
the validation check individually for these two outcome procedures, as one way being valid, does
not imply the other is. E.g. in the following code, the assertion can be removed in the
forward run, but not the backwards, as giving an uneven number to \texttt{double} backwards will
result in information loss due to integer division.

\begin{lstlisting}[language=C]
    procedure double(int bit)
    {
        local int z = bit
        bit += z
        delocal int z = bit / 2
    }
\end{lstlisting}

\subsubsection{Interacting with the \texttt{z3} API} \label{sec:z3interaction}
The \texttt{z3} API has bindings to several general purpose programming languages \cite{z3:api}.
As this project involves prototyping a compiler using \texttt{z3} to statically analyze a program,
the language used should allow quick prototyping, and easy tree structure manipulation. For this
purpose \texttt{Haskell} has been chosen. It allows for easy tree manipulation using pattern
matching, and it's type construct \texttt{data}. It's functional nature, and the fact that it is
statically typed, also makes it easy to quickly prototype functionality.
\\
\\
\texttt{Haskell}'s interface with \texttt{z3} is build using the \texttt{C} API as it's fundament
\cite{z3:api2}. In particular this bachelor project uses the \text{Z3.Monad} wrapper.
Which puts abstraction on top of the base level of the API, allowing for easier manipulation
and creation of the \texttt{z3} model.


