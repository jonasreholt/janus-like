\documentclass[11pt]{article}
\usepackage[a4paper, total={7in, 9in}]{geometry}
\usepackage[utf8]{inputenc}
\usepackage{amsmath, amssymb, stmaryrd}
\usepackage{hyperref}
\usepackage{caption, subcaption}
\usepackage{pdfpages}
\usepackage{float}
\usepackage{listings, color}
\usepackage{enumitem}
\usepackage{fancyhdr}
\usepackage[nottoc,numbib]{tocbibind}       % Including bibtex references in table of content
\usepackage{cleveref}
\usepackage[most]{tcolorbox}
\usepackage{varwidth}
\usepackage{multirow}
\usepackage{multicol}

% matthis.kruse@cispa.de

%%%%%%%%%%%%%%%%%%%%%%%%%%%%%%%%%%%
%    Commands for ease of use     %
%%%%%%%%%%%%%%%%%%%%%%%%%%%%%%%%%%%
\newcommand{\versionnr}{0.6}
\renewcommand{\*}{\cdot}                    % a pretty multiplication sign
\renewcommand{\implies}{\rightarrow}        % implies now is simple right arrow: -->
\newcommand{\lan}{\texttt{Japa} }
\newcommand*\lsin{\lstinline[columns=fixed]} % inline code blocks
\newcommand{\assigneq}{\oplus{\kern-5.0pt}=} % operator=

% Commands used during production
\newcommand{\rr}{(\textbf{\emph{REVIEW READY}})}
\newcommand{\dn}{(\textbf{\emph{DONE}})}
\newcommand{\ms}{(\textbf{\emph{MISSING}})}

%%%%%%%%%%%%%%%%%%%%%%%%%%%%%%%%%%%
%   Defining definition boxes     %
%%%%%%%%%%%%%%%%%%%%%%%%%%%%%%%%%%%
\newtcbtheorem[number within=section]{Definition}{}
{
        enhanced,
        sharp corners,
        attach boxed title to top left={
            xshift=-1mm,
            yshift=-5mm,
            yshifttext=-1mm
        },
        top=1.5em,
        colback=white,
        colframe=black,
        fonttitle=\bfseries,
        boxed title style={
            sharp corners,
            size=small,
            colback=black,
            colframe=black,
        } 
}{def}

\newenvironment{myDefinition}[2]{ \begin{Definition}[adjusted title=#1]{}{#2} 
    \textbf{Definition \thetcbcounter.} }{\end{Definition}}

%%%%%%%%%%%%%%%%%%%%%%%%%%%%%%%%%%%
%   Defining Theorem boxes        %
%%%%%%%%%%%%%%%%%%%%%%%%%%%%%%%%%%%
\newtheorem{theorem}{Theorem}[section]

%%%%%%%%%%%%%%%%%%%%%%%%%%%%%%%%%%%
% Defining a beautiful code block %
%%%%%%%%%%%%%%%%%%%%%%%%%%%%%%%%%%%
\definecolor{codegreen}{rgb}{0,0.6,0}
\definecolor{codegray}{rgb}{0.5,0.5,0.5}
\definecolor{codepurple}{rgb}{0.58,0,0.82}
\definecolor{backcolour}{rgb}{0.95,0.95,0.92}
\definecolor{bluekeywords}{rgb}{0.13,0.13,1}
\definecolor{greencomments}{rgb}{0,0.5,0}
\definecolor{redstrings}{rgb}{0.9,0,0}

\lstdefinestyle{mystyle}{
    backgroundcolor=\color{backcolour},
    commentstyle=\color{codegreen},
    keywordstyle=\color{magenta},
    numberstyle=\tiny\color{codegray},
    stringstyle=\color{codepurple},
    basicstyle=\ttfamily\footnotesize,
    breakatwhitespace=false,
    breaklines=true,
    captionpos=b,
    keepspaces=true,
    numbers=left,
    numbersep=5pt,
    showspaces=false,
    showstringspaces=false,
    showtabs=false,
    tabsize=2
}

\lstset{style=mystyle}

% CSS
\lstdefinelanguage{CSS}{
  keywords={color,background-image:,margin,padding,font,weight,display,position,top,left,right,bottom,list,style,border,size,white,space,min,width, transition:, transform:, transition-property, transition-duration, transition-timing-function},	
  sensitive=true,
  morecomment=[l]{//},
  morecomment=[s]{/*}{*/},
  morestring=[b]',
  morestring=[b]",
  alsoletter={:},
  alsodigit={-}
}

% JavaScript
\lstdefinelanguage{JavaScript}{
  morekeywords={typeof, new, true, false, catch, function, return, null, catch, switch, var, if, in, while, do, else, case, break},
  morecomment=[s]{/*}{*/},
  morecomment=[l]//,
  morestring=[b]",
  morestring=[b]'
}

%%%%%%%%%%%%%%%%%%%%%%%%%%%%%%%%%%%
%        Defining header          %
%%%%%%%%%%%%%%%%%%%%%%%%%%%%%%%%%%%
\pagestyle{fancy}
\lhead{\footnotesize Static Assertion Checking Optimization in a Janus-like
    programming language - version \versionnr}
\rhead{\today}


\begin{document}

%%%%%%%%%%%%%%%%%%%%%%%%%%%%%%%%%%%
%       Front page print          %
%%%%%%%%%%%%%%%%%%%%%%%%%%%%%%%%%%%
\includepdf[pages=1]{../../bachelor_project_description}
\newpage

\section*{Abstract}
A reversible programming language forces the programmer to write reversible programs, i.e.
a program that can run in both direction; computing both from input to output, and output to
input. Reversible language are interesting to study, as it removes the theoretical lower limit
of heat generation, and that one only needs to write a single program, to handle functionality
that has a natural inverse, e.g. encryption. One downside to reversible language is, the added
overhead needed to guarantee correct behavior in both directions.

This thesis presents a optimizer module aimed at removing this runtime overhead. As the overhead
stem from runtime assertions, the presented optimizer tries to prove these assertions to always
hold using the satisfiability modulo  theory solver \texttt{z3}. To do this the thesis first
presents a simple imperative reversible language, similar to \texttt{Janus}. The optimizer
translates from this languages abstract syntax tree into a \texttt{z3} model, which is then
queried at every program point, that involves a runtime assertion. This thesis addresses some
of the complexion of this translation and querying to \texttt{z3}, when going from a reversible
language, e.g. the user defined possibly bogus runtime assertions.
\newpage
\tableofcontents
\newpage

% Motivation, problem, mini-example, boundaries

% Problem definition
% Boundaries
% What is reversable languages?
\section{Introduction \ms}
%% MISSING STUFF
%   1. Find references
%   2. What is missing
%   3. Overview of methods used
%   4. Maybe a teaser with a Japa code example
Reversible programming languages allow the programmer to write programs, that can be run
both forward and backwards. Getting this ability into a language impacts the basic structure
such that assignments, conditional operations, and loops must be handled in another way
than traditional languages do. The reversibility of the programs does however have the benefits
of:

\begin{itemize}
    \item Lowering side-channel attack, as it does not constantly delete memory to make room for
    new information [FIND CITATION], thereby creating programs that generates a more constant
    stream of heat.

    \item Removing the theoretical lower limit of heat generation [FIND CITATION], making way
    for lowering the energy usage of computers (This does however require computers build
    for reversibility).

    \item Models the world of physics more precise, as physics in itself is reversible
    [FIND CITATION].

    \item Only needing to write one program, when functionality such as zip/unzip, that has
    a natural inversion that you want. This allows the programmer to write one program, prove
    correctness of one program, and then ship "two" programs.
\end{itemize}
\noindent
This project focuses on the last item for two reasons: 1) that the majority of computers today are
not reversible, and 2) that translating from a reversible language to an irreversible
generates certain overhead, as i.e. \texttt{if}-statements need both an entering condition and an
exiting assertion for reversibility; meaning the program needs to check an extra assertion each
time an \texttt{if}-statement is run. This assertion is ends up taking approximately 7 instruction,
containing a conditional jump removing linearity from the code, when translating to
\texttt{x86-64} assembly code. E.g. the simple dummy function:

\begin{lstlisting}[language=c++]
    void f(int a)
    {
        assert(a == 0);
    }
\end{lstlisting}
\noindent
Gets the assertion translated, using gcc version 11.2, into:

\begin{lstlisting}[language={[x86masm]Assembler}]
    ...
    cmp     DWORD PTR [rbp-4], 0
    je      .L3
    mov     ecx, OFFSET FLAT:.LC0
    mov     edx, 5
    mov     esi, OFFSET FLAT:.LC1
    mov     edi, OFFSET FLAT:.LC2
    call    __assert_fail
.L3:
    ...
\end{lstlisting}
\noindent
Meaning 7 instructions could be optimized away, herein including a conditional jump,
that requires correct jump prediction to run efficiently.

\subsection{Limitations \ms}
The focus of this project is on finding whether a theorem prover is available for making static
program analyse in compile time, checking whether these extra assertions can be removed. Hence
there will be no other focus on optimization in the code generation
\\
\\
Also as this is a bachelor thesis the focus will be on freely available theorem provers.

\subsection{Acknowledgements \ms}
Special thanks to [People who help prrof read] and Matthis Kruse
Robert Glück.
\section{Janus-like language definition}
% Definition of chosen language design and theory behind reversability
\section{Theorem prover}
% Find representative set of theorem provers, and analyze in context of project
\section{Compiler Structure}
% Overview of compiler structure with image
% Information and analysis of choice of programming language

\subsection{Lexer}
% Theory behind lexing and the implementation

\subsection{Parser}
% Theory behind parser and the implementation

\subsection{Optimization}
% Strategy for optimization
% Integration with theorem prover
Because the compiler translates directly from the source language into an abstract syntax tree,
and then into \texttt{C++}, the optimization must be done on the abstract syntax tree, as it is
the easiest data structure to work on of the three. Doing the optimization directly on the source
language itself, saves the computation of translating to some intermediate language, but does
make this optimizer local to the source language.
\\
\\
This optimization involves four steps:
\begin{enumerate}
    \item Discovering language constructs, that introduce assertions on translation.
    \item Locating and gathering information for the prove.
    \item Translating subpart of abstract syntax tree into \texttt{z3}.
    \item Deciding on whether to optimize the assertion or not, based on the answer
          from \texttt{z3}.
\end{enumerate}

\subsubsection{Language Constructs Introducing Asseritons}
The first point can be discovered by consulting the language specification introduced
in section \ref{sec:language-def}. Here it is imminent, that the following construct introduce
assertions for the translated code:
\begin{itemize} % TODO: Give example showing where assertions are created
    \item Local declaration as they require deallocation at some point for reversibility.
    \item If statement as the program needs to know whether the if-path was chosen or not when
          reversing computation.
    \item from statement as the first boolean expression following \texttt{from} can only be true
          before the loop runs, and the expression after \texttt{until} can only be true after
          all iterations are complete.
\end{itemize}
\noindent
Which means that when running through the abstract syntax tree, the compiler must pause translation
when the above constructs are found, and then the optimizer module must be called to tell
the translator, whether to include assertions or not.

\subsubsection{Gathering Information for \texttt{Z3}}
Point two has two main obstacles: 1) to know how much information \texttt{z3} needs for the proof,
and 2) granting the optimizer access to this appropriate subtree.

For 1) a simply approach could be to give the tree representing the current procedure to the
optimizer. This could give too much information in the sense that only part of this subtree is
actually needed to perform the validation check. e.g. in the small program below the only thing
needed for the validation is the two last lines. However, this would require backtracking the
tree while keeping a list of "unassigned" variables, until this list is empty, so the first
method is used in this optimizer. For future work, it would be a good idea to check whether the
other approach would be faster. The below piece of code also reveal two other problem in regards
to information: Global variables and function calls.

Getting the value of a global variable
requires that the language be interpreted on the go, making the question a matter of a table
lookup; there is however no interpretation going on in this compiler, limiting the optimizer
into using unspecified variables for globals.

Function calls hide information that might be needed when performing validation for a procedure.
To mitigate this, aggressive inlining is used. This does however pose the question of recursive
procedures. I will leave this question for now, and return to it below, when addressing loops.

\begin{lstlisting}[language=C++]
    int c

    procedure g()
    {
        c += 5
    }

    procedure f(int a)
    {
        call g()
        a += c
        local b = 1
        delocal b == 1
    }
\end{lstlisting}

\subsubsection{Translating to \texttt{Z3}} % and deciding on optimizing or not!
As the compiler is implemented in \texttt{Haskell} these bindings for \texttt{z3} in
\texttt{Haskell} is used \cite{Z3:BINDINGS}. The optimizer module is therefore a translator from
the abstract syntax tree into these bindings, and can be seen in the \nameref{sec:appendices}.
\\
\\
Validating whether an assertion can be removed with \texttt{z3}, can be done by creating a model
of the code, that ends up inverting the boolean expression being asserted, and then checking
for satisfiability. If this is not satisfiable, then the assertion can never be false, and
can safely be removed.

Because of constraints in \texttt{z3} the following language constructs of \texttt{JAPA} is
problematic:
\begin{itemize}
    \item Moderating statements.
    \item Conditional statements.
    \item Loops.
    \item Function calls.
    \item Reversibility.
\end{itemize}
\noindent
For solving the difficulties of these language constructs, I have used the approach outlined
in the talk \cite{Z3:TRANSLATION} at Compose 2016.

\subsubsection*{Moderating Statements}
The problem with this construct is that all variables in \texttt{z3} are immutable. This can
be addressed by creating new fresh variables, and asserting these fresh variables to the moderated
value e.g.
\begin{verbatim}
    x += 5  =>  (declare-const x1 Int)
                (assert (= x1 (+ x 5)))
\end{verbatim}

\subsubsection*{Conditional statements}
If the assertion being validated is not associated to a specific conditional, the program will
not know which path is going to be taken at run time. Therefore both paths must be constructed,
and then the \texttt{ite} function from \texttt{z3} can be used to determine the path. e.g.
\begin{verbatim}
    local int x = 5         (define-fun x () Int 5)
    if (x < 5)              (declare-const x1 Int)
    {                   =>  (assert (= x1 (- x 5)))
        x -= 5              (declare-const x2 Int)
    } fi (x == 0)           (assert (= x2 (ite (< x 5) x x1)))
    delocal int x == 5      (assert not (= x2 5))
\end{verbatim}

\subsubsection*{Loops}
Because \texttt{z3} does not have a construction for loops, loops can only be validated by unrolling
them. The question of how far to unroll loops is however rather complex. e.g. what should be done
with non-terminating loops, and how to detect these? Also how far should loops be unrolled? One
important factor is, that the optimizer must be conservative, so to not change the runtime behavior.
Which means if it's not possible to analyze how many times to unroll, the optimization is blocked
by loops.

When it comes to \texttt{for}-loops the analysis is relatively simple, as the constructs both
tell a starting value, what happens to this variable at each iteration, and when to stop.
So as long as the local variable being declared in the beginning, is the same as the one being
deallocated after the loop, and the same being moderated at each iteration, it is straight forward.
However if one of these are not the case, the analysis becomes harder.

The unrolling of a simple \texttt{for}-loop can be done by transforming it into a series of
\texttt{if}-statements that can be represented in \texttt{z3}:
\begin{verbatim}
    local Int bit = 1                   local int bit = 1
    for int i = 0                       local int i = 0
    {                                   if (!(i == 10))
        local int z = bit           =>  {
        bit += z                            local int z = bit   
        delocal int z = bit / 2             bit += z
    } i += 1; untill (int i = 10)           delocal int z = bit / 2
                                            i += 1
                                            if (!(i == 10))
                                            {
                                                ...
                                            } fi (i == 10)
                                        } fi (i == 10)
                                        delocal int i = 10
                                        delocal int bit = 2048
\end{verbatim}

% analyzing more complex end conditions, and show example with from loop!
% Maybe use https://web.njit.edu/~mili/ccb.pdf for removing end assertion from until
% Probably just end at unrolling n amount of time, and then hoping n is large enough.

% \begin{verbatim}
%     from (bit == 1) loop            if (bit == 1 && !(bit * bit > num))
%     {                               {
%         local int z = bit               local int z = bit
%         bit += z                =>      bit += z
%         delocal int z = bit / 2         delocal int z = bit / 2
%     } until ((bit * bit) > num)         if (!(bit * bit > num))
%                                         {                                    
%                                             local int z = bit
%                                             bit += z
%                                             delocal int z = bit / 2
%                                             if (!(bit * bit > num))
%                                             {
%                                                 ...
%                                             } fi (bit * bit > num)
%                                         } fi (bit * bit > num)
%                                     } fi (bit * bit > num)
% \end{verbatim}

\subsubsection*{Function Calls}
% TODO: Talk about recursion!!!
To model the functions correctly in \texttt{z3} inlining must be done, as all changes done
by a function is through side effects. After inlining the function, it can be modeled using the
other strategies outlined above.

\subsubsection*{Reversibility}
Because the translation is from a reversible language to a irreversible language, every procedure
will be translated into two: One going forward, and one backwards. It is necessary to perform
the validation check individually for these two outcome procedures, as one way being valid, does
not imply the other is. E.g. in the following code, the assertion can be removed in the
forward run, but not the backwards, as giving an uneven number to \texttt{double} backwards will
result in information loss due to integer division.

\begin{lstlisting}[language=C]
    procedure double(int bit)
    {
        local int z = bit
        bit += z
        delocal int z = bit / 2
    }
\end{lstlisting}

% Problem with language constructs
% \begin{itemize}
%     \item if-else statements. %do the ite thing if we don't know we take it otherwise assert if
%     \item loops. % unroll
%     \item function calls. %Inline? unroll recursion?
%     \item Reversibility % do the prove forward and backward, one way is not enough bc e.g.janus playground example with square
% \end{itemize}


\subsection{Translation to \texttt{C++}}
% Theory behind the optimization and implementation
\section{Testing of Compiler \ms}
% Benchmark of compiled code: optimized vs. non-optimized
% Benchmark of compiler with and without optimization
\section{Benchmark of Compiled Code \ms}
\section{Packing Compiler}

\subsection{Compiler Package and Usage}
% Installation, requirements, usage

\subsection{Web Interface for Compiler}
\section{Conclusion and future work }
This project presented the language \lan derived from the reversible language \texttt{Janus}.
\lan keeps the reversibility property derived from \texttt{Janus}, but changes e.g.\ the loop
construct in order to remove some runtime assertions directly from the language construct.

The \lan compiler presented in this project, translates from the source language into \texttt{C++}
code. It tries to remove runtime assertions by the use of the SMT solver \texttt{z3}. This is
done by recursing through the AST, translating into a \texttt{z3} model using its API for
\texttt{Haskell}. Each time a construct creating a runtime assertions is met, a query is made to
\texttt{z3}, in order to determine whether the assertion can be proven to always hold. This is
done by the steps:
\begin{enumerate}
      \item If it is a user-generated assertion, the assertion is checked to be satisfiable.
      \item The assertion is negated, and the \texttt{z3} model is checked to be satisfiable.
      \item If the negated assertion is not satisfiable, it means the assertion is always valid
            and can be optimized away.
\end{enumerate}
\noindent
In order to create a proper translation into \texttt{z3}, a simple type checker goes through the
AST before the optimization phase, both in order to (of cause) type check, but more importantly
annotate the AST with both types and array sizes.
\\
\\
Using the SMT solver \texttt{z3} the \lan compiler showed ability, to optimize most runtime
assertions away. The area in which the chosen method proved to be less effective, was when
analyzing reversed procedures. This probably stems from the fact, that the reversed procedure
some times work with "unspoken" starting contracts e.g.\ that the input numbers are always equal
when the forward procedure has been executed before.

Loop constructs also showed resistance towards optimizations, which makes sense, as \texttt{z3}
has no theory concerning loops. This means measures needed to be taken, in order to gather as much
information as safely possible, for \texttt{z3} to correctly validate assertions. In the best case
scenario loops were unrollable, so the pure information was retrievable, but in all other
cases, generalization methods was needed, to remove any possibly wrong information from the
SMT model.

This project showed great promise in regards to optimizing assertions away from imperative
reversible languages, but at present moment, the execution time of the compiler is rather slow.
For a simple Fibonacci program consisting of $21$ lines of code, the compile time was
$6.2$ seconds, with optimizations. This slow speed stem from the use of the \texttt{z3}
solver, and is probably caused by bad usage by the writer of this bachelor thesis.
\\
\\
The \lan language, as presented in this project, is rather simple, with simple types and
no recursion. For future work, implementing recursion would make sense, in crating a more
modern and pleasant language to work with. The slow execution time of the compiler should,
however, be a top priority, to improve development iteration time, to actually make the
optimization useful, as a "sparring" partner, to improve speed of developed code. As previously
mentioned the focus for this, should probably start at the use of \texttt{z3}, mainly of which
theories are called with the solver.

To avoid removing previously known information about variables, when analyzing loops that cannot
be unrolled, a method somewhat like the one presented in \cite{ai}, where loops are analyzed
using a combination of abstract interpretation and a SMT. This might be able to improve the
analysis of loops.


\subsection{Reflection over project }
After having completed this project, and experiencing the consequences of the workflow I
adopted, I can see that automated tests should probably have been implemented from the start.
The test section also reveals, the lack of unit tests for each function, as only the full
behavior of the compiler is tested. I fell in the trap, of ignoring setting up these tests
using \texttt{cabal}, as I simply felt swarmed already having to learn \texttt{Haskell}, \texttt{z3},
and all the other things needed for this thesis. I probably shouldn't have ignored it, as it lead
to some annoying bugs, when I finally implemented the tests of the full behavior.

As an extension to this, it would probably have been nice to work with more people on this
project, so I had someone to spar with on a daily basis.
\\
\\
Looking back it would probably have made sense to use \texttt{Boogie 2} instead of directly
communicating with \texttt{z3}. \texttt{Boogie 2} is a intermediate language representation, used
for verifying programming languages. Hence it works for multiple SMT solvers, and is used in
projects such as \texttt{Dafny}. Working with this intermediate language, might have made
it easier correctly modelling program structures. The reason why I did not chose it was, that
\lan is a reversible language, meaning it does not follow the ordinary program structure, meaning
representing the reversible semantics in the intermediate code might be unnecessary complicated.

\newpage
\bibliography{references}
\bibliographystyle{ieeetr}

\newpage
\section{Appendices} \label{sec:appendices}
% Test programs, test data, code, etc.
\subsection{Source Code}
\subsubsection{Main Function}
\lstinputlisting[language=Haskell]{../app/Main.hs}

\subsubsection{Syntax}
\lstinputlisting[language=Haskell]{../app/Syntax.hs}

\subsubsection{Parser}
\lstinputlisting[language=Haskell]{../app/Parser.hs}

\subsubsection{Type Checker}
\lstinputlisting[language=Haskell]{../app/TypeCheckAnnotate.hs}

\subsubsection{AST Reversing}
\lstinputlisting[language=Haskell]{../app/AstReversing.hs}

\subsubsection{Evaluating Constant Expressions}
\lstinputlisting[language=Haskell]{../app/EvalExpr.hs}

\subsubsection{Optimizer}
\lstinputlisting[language=Haskell]{../app/AssertionRemoval.hs}

\subsubsection{AST Renaming}
\lstinputlisting[language=Haskell]{../app/RenameProcedures.hs}

\subsubsection{\texttt{C++} Code Generation}
\lstinputlisting[language=Haskell]{../app/JapaToCpp.hs}

\subsubsection{Running tests}
\lstinputlisting[language=bash]{../bin/tester.sh}
\begin{figure}
    \centering
    \includegraphics{imgs/testrun.png}
    \caption{A run through of the different gold file tests.}
    \label{fig:testrun}
\end{figure}

\subsubsection{Makefile}
\lstinputlisting[language=bash]{../Makefile}

\subsubsection{Benchmark script}
\lstinputlisting[language=Bash, label={lst:runbenchmark}]{../benchmarks/runbenchmark.sh}

\subsubsection{Benchmark programs} \label{sec:benchmark-programs}
\lstinputlisting[language=c++]{../web/examples/factorial.japa}
\lstinputlisting[language=c++]{../web/examples/fib.japa}
\lstinputlisting[language=c++]{../web/examples/perm-to-code.japa}
\lstinputlisting[language=c++]{../web/examples/unrollableLoop-invariant.japa}

\end{document}

% Focus on the report now
% Interesting things are whatever goes beyond IPS course curriculum
% The uninteresting things needs to be mentioned, but no theory discussion for it!!

% In frontend only mention unconventional thing

% 4.3:
% Why on earth am I reversing the AST
% And how am I reversing (recursive descend)
%   How it happens precisely
% Setting it together with the overall approach of removing assertions.

% Use more code snippet in everything "new" e.g. loops


%%%%%%%%%%%%%%%%%%%%%%%%%%%%%%%%%%%%%%%%%%%%%%%%%%%%%%%%%%%%%%%%%%%%%%%%%%
% Internal trial exam where I just present the presentation, to try out the timing

% Exam room: HCØ A106 "A106 at HCØ for the thesis defenses on June 21"

% Benchmark:
%            If one had more examples one could say: In this "diverse" set of programs, that
%            cover the different analysis cases, it works like this for nested conditionals,
%            and so forth. So it should be more about analytical overview of what it can do
%            and what it cannot, and not so much about runtime. So highlight some cases where
%            it could not be removed, show that it possibly could be easily fixed.
%           So ite, for unrollable, for generalized, nested conditional and so forth.
%           Did the unroll work well, do my
%           generalization work? Can stuff be "fixed" easily by handcrafted assertions.


% Maybe add a flag to chose loop bound. Add a micropoint to why 100 was used as default

% 3.2.2: Either expand it with concrete examples of the API, or mix it with another section.