\section{Reflection over project }
After having completed this project, and experiencing the consequences of the workflow I
adopted, I can see that automated tests should probably have been implemented from the start.
The test section also reveals, the lack of unit tests for each function, as only the full
behavior of the compiler is tested. I fell in the trap, of ignoring setting up these tests
using \texttt{cabal}, as I simply felt swarmed already having to learn \texttt{Haskell}, \texttt{z3},
and all the other things needed for this thesis. I probably shouldn't have ignored it, as it lead
to some annoying bugs, when I finally implemented the tests of the full behavior.

As an extension to this, it would probably have been nice to work with more people on this
project, so I had someone to spar with on a daily basis.
\\
\\
Looking back it would probably have made sense to use \texttt{Boogie 2} instead of directly
communicating with \texttt{z3}. \texttt{Boogie 2} is a intermediate language representation, used
for verifying programming languages. Hence it works for multiple SMT solvers, and is used in
projects such as \texttt{Dafny}. Working with this intermediate language, might have made
it easier correctly modelling program structures. The reason why I did not chose it was, that
\lan is a reversible language, meaning it does not follow the ordinary program structure, meaning
representing the reversible semantics in the intermediate code might be unnecessary complicated.