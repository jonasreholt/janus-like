% Motivation, problem, mini-example, boundaries

% Problem definition
% Boundaries
% What is reversable languages?
\section{Introduction }
%% MISSING STUFF
%   3. Overview of methods used
%   4. Maybe a teaser with a Japa code example
Reversible programming languages allow the programmer to write programs, that can be run
both forward and backwards. Getting this ability into a language impacts the basic structure
such that assignments, conditional operations, and loops must be handled in another way
than traditional languages do. The reversibility of the programs does however have the benefits
of:

\begin{itemize}
    \item Lowering side-channel attack, as it does not constantly delete memory to make room for
    new information \cite{hermes}, thereby creating programs that generates a more constant
    stream of heat.

    \item Removing the theoretical lower limit of heat generation \cite{article:3}, making way
    for lowering the energy usage of computers (This does however require computers build
    for reversibility).

    \item Models the world of physics more precise, as physics in itself is reversible.

    \item Only needing to write one program, when functionality such as zip/unzip, that has
    a natural inversion that you want. This allows the programmer to write one program, prove
    correctness of one program, and then ship "two" programs.
\end{itemize}
\noindent
This project focuses on the last item for two reasons: 1) that the majority of computers today are
not reversible, and 2) that translating from a reversible language to an irreversible
generates certain overhead, as i.e. \texttt{if}-statements need both an entering condition and an
exiting assertion for reversibility; meaning the program needs to check an extra assertion each
time an \texttt{if}-statement is run. This assertion is ends up taking approximately 7 instruction,
containing a conditional jump removing linearity from the code, when translating to
\texttt{x86-64} assembly code. E.g. the simple dummy function:

\begin{lstlisting}[language=c++]
    void f(int a)
    {
        assert(a == 0);
    }
\end{lstlisting}
\noindent
Gets the assertion translated, using gcc version 11.2, into:

\begin{lstlisting}[language={[x86masm]Assembler}]
    ...
    cmp     DWORD PTR [rbp-4], 0
    je      .L3
    mov     ecx, OFFSET FLAT:.LC0
    mov     edx, 5
    mov     esi, OFFSET FLAT:.LC1
    mov     edi, OFFSET FLAT:.LC2
    call    __assert_fail
.L3:
    ...
\end{lstlisting}
\noindent
Meaning 7 instructions could be optimized away, herein including a conditional jump,
that requires correct jump prediction to run efficiently.

\begin{itemize}
    \item Done at soruce program level (Also why not on IL level)
    \item Not possible to do on assembly level
\end{itemize}

\subsection{Limitations }
The focus of this project is on finding whether a theorem prover is available for making static
program analyze in compile time, checking whether these extra assertions can be removed. Hence
there will be no other focus on optimization in the code generation
\\
\\
The compiler created for this thesis, does not implement recursive procedures. In fact it also
does not catch the error, of someone writing a recursive procedure. At the moment this is "caught"
by the optimizer spiralling in a endless loop, either until the stack runs out, or until the
runtime exceeds the timeout limit set to $60$ seconds.
\\
\\
Also as this is a bachelor thesis the focus will be on freely available theorem provers.

\subsection{Acknowledgements }
A special thanks to my advisor Matthis Kruse and supervisor Robert Glück, for helping me through
the process of implementing the compiler using the SMT \texttt{z3}, and for helping in regards to
the theory behind reversible languages, and different techniques for statically optimizing
programming languages.

Also thanks to all the people who helped proofread sections of this report, giving valuable
insight and feedback on the material. It would have been filled with a lot more grammatical errors
and the like, had it not been for them.