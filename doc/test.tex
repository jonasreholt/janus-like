\section{Testing of compiler } \label{sec:tests}
%Also add the example of failing test
Testing of the compiler is done in two ways: Automatic and manually. Automatic tests is
done by running the compiler, inspecting the output code, running the output using \texttt{g++},
manually writing the prober model in \texttt{z3}, and then if it all works, a gold file is
created with the output code, and the test then checks the output of the compiler against this
gold file. This method is used, to track any regression in the code output. This is done, as
the main function of the compiler is to optimize the code.

Manual tests were conducted during production of the different modules. This simply involved
running the code in \texttt{ghci}, and printing out each step of the way; prototyping towards
the end compiler.
\\
\\
The automatic tests are run using the bash script from \autoref{lst:tester}, which is created
on the basis of one handed out during the course "Computer Systems" at Copenhagen University
\cite{compSys}. The test
script goes through every test \lan file, and compares the output with the corresponding
gold file. \lsin{2>&1} is used to direct both standard in and standard out from the compiler
run, so that negative tests, can also be created. If the two files match a green text
is written with "Success".

There are $32$ test programs that cover a small set of edge cases. Currently all test programs
are successful. An image of a test-run can be seen in \autoref{fig:testrun}.


% Talk about how testing could have been improved
\begin{lstlisting}[language=Bash, label={lst:tester}]
...
compare () {
    if [ -f $2 ]; then
        local expected="$(cat $2)"
        if [[ "$output" != "$expected" ]]; then
          echo "Output for $0 does not match gold file."
          echo "Compare $1 with $2."
          return 1
        else
            return 0
        fi
    fi
    echo "gold file ($2) not present"
    return 1
}

...

for f in $test_dir/*; do
    fname="$(basename "$f")"
    program_name="$(echo $fname | sed 's/.japa$//')"
    printf "%*s" $file_len " $fname:  "

    # Save both stderr and stdout in output, so negative tests can also be tested
    output=$($compiler $f 2>&1)
    if compare $fname "$gold_dir/$program_name.gold"; then
       echo -e "\033[92mSucess.\033[0m"
    else
       echo -e "\033[91mTest error.\033[0m"
    fi
done
\end{lstlisting}